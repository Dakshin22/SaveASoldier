\documentclass[conference]{IEEEtran}
\IEEEoverridecommandlockouts
% The preceding line is only needed to identify funding in the first footnote. If that is unneeded, please comment it out.
\usepackage{cite}
\usepackage{amsmath,amssymb,amsfonts}
\usepackage{algorithmic}
\usepackage{graphicx}
\usepackage{textcomp}
\usepackage{xcolor}
\def\BibTeX{{\rm B\kern-.05em{\sc i\kern-.025em b}\kern-.08em
    T\kern-.1667em\lower.7ex\hbox{E}\kern-.125emX}}
\begin{document}

\title{Title of Paper}

\author{Name}

\maketitle

\begin{abstract}
Brief overview of what you did, why it's important and results.
\end{abstract}

\section{Introduction}
\textcolor{red}{Write about what you are doing and why it is important. Need to discuss related work on pain recognition here. Undergraduates need at least
5 papers referenced and discussed. Graduate students need at least 10. Just say who did it and a general overview of what they did. Google scholar works well for this.
NOTE: It does not have to be physiological. Can be face, action units, etc.}
\section{Method}
\textcolor{red}{Discuss random forest here. Cite Brieman's work - can find reference in Google Scholar. Talk about the fusion approach you used. If you did 
the extra credit detail both of them.}
\section{Experiments and Results}
\textcolor{red}{What experiments did you do and what results did you get? Why do you think you got them?}
\section{Discussion and Conclusion}
\textcolor{red}{Do you think phsyiological data is good for pain recognition? What about fusion? Is there a better approach?}
\section{Future Work}
\textcolor{red}{This section is for \textbf{Graduate Students Only}. Come up with a new approach for pain recognition. What future work can you do with what 
you've done here and learned in the class?}
\bibliographystyle{ieee}
\bibliography{references}
\end{document}
